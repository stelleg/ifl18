\section{Introduction} \label{sec:introduction}
Non-strict languages, such as Haskell, rely heavily on call-by-need semantics to
ensure efficient execution. Without the memoization of results provided by
call-by-need, Haskell would be prohibitively inefficient, often exponentially
slower than its call-by-value counterparts. Haskell is one of the purest
languages in wide use for software development, making reasoning about
correctness in Haskell easier than most languages. It is for this reason that
we would like to have a compiler that gives formal guarantees about preservation
of call-by-need semantics. We wish to ensure that any reasoning we do about
our non-strict functional programs is preserved through compilation, and that
their execution will not be prohibitively inefficient.

Unfortunately, one of the challenges for formalization of non-strict compilers
is that the semantics of call-by-need abstract machines tend to be complex,
incorporating complex optimizations into the semantics, requiring preprocessing
of terms, and closures of variable sizes \cite{jonesstg, TIM}. Recently we
developed a particularly simple abstract machine for call-by-need, the
$\mathcal{CE}$ machine \cite{cem}. In addition to being exceedingly simple to
implement and reason about, the machine shows performance often comparable to
state-of-the-art.

Verified compilers provide powerful guarantees about the code they generate and
its relation to the corresponding source code \cite{chlipala2007certified,
leroy2012compcert, cakeml14}.  In particular, for higher order functional
languages, they can often ensure that the non-trivial task of compiling lambda
calculus and its derivatives to machine code is implemented correctly,
preserving source semantics. The amortized return on investment for verified
compilers is high: any reasoning about any program which is compiled with a
verified compiler is provably preserved. 

Existing verified compilers have focused on call-by-value semantics
\cite{chlipala2007certified, leroy2012compcert, cakeml14}. This semantics has
the property of being historically easier to implement than call-by-need, and
therefore likely easier to reason about formally. In this paper, we build on
recent work developing a simple method for implementing call-by-need semantics,
which has enabled us to implement, and reason formally about the correctness of,
call-by-need. We use the Coq proof assistant \cite{barras1997coq} to implement
and prove the correctness of our compiler. We start with a source language of
$\lambda$ calculus with deBruijn indices:
\begin{align*}
 t &::= t \; t \; | \; x \; | \;  \lambda \; t \\
 x &\in \mathbb{N}
\end{align*}
Our source semantics is the big-step operational semantics of the $\mathcal{CE}$
machine, which uses shared environments to share results between instances of a
bound variable. To strengthen the result, and relate it to a better-known
semantics, we also show that the call-by-name $\mathcal{CE}$ machine implements
Curien's call-by-name calculus of closures. 

It may surprise the reader to see that we do not start with a better known
call-by-need semantics; we address this concern in
Section~\ref{sec:discussion}.  We hope that the proof of compiler correctness,
along with the proof that our call-by-name version of the semantics implements
Curien's call-by-name semantics, convinces the reader that we have indeed
implemented a call-by-need semantics, despite not using a better definition of
call-by-need. 

For our target, we define a simple instruction machine, described in
Section~\ref{sec:im_semantics}. This simple target allows us to describe the
compiler and proofs concisely for the paper, while still allowing
flexibility in eventually verifying a compiler down to machine code for some
set of real hardware, e.g., x86, ARM, or Power. 

Our main results is a proof that whenever the source semantics evaluates to a
value, the compiled code evaluates to the same value. While there are stronger
definitions of what qualifies as a verified compiler, we argue that this is
sufficient in Section~\ref{sec:discussion}. This main result, along with the
proof that the call-by-name version of our semantics implements Curien's
calculus of closures, are the primary contributions of this paper. We are
unaware of any existing verified non-strict compilers, much less a verified
compiler of call-by-need. 

The paper is structured as follows. In Section~\ref{sec:background} we give the
necessary background. In Section~\ref{sec:cem_big} we describe the source syntax
and semantics (the big-step $\mathcal{CE}$ semantics) in detail.  We also use
this section to define a call-by-name version of the semantics, and
show that it implements Curien's calculus of closures \cite{curien1991abstract}.  In
Section~\ref{sec:cem_small} we describe the small-step $\mathcal{CE}$ semantics
and its relation to the big-step semantics. In Section~\ref{sec:im_semantics}
we describe the instruction machine syntax and semantics. In
Section~\ref{sec:compiler} we describe the compilation from machine terms to
assembly language. In Section~\ref{sec:correctness} we describe how evaluation
of compiled programs is related to the small-step $\mathcal{CE}$ semantics. the
compilation of terms into assembly language, and the instruction machine
semantics, and finally compose this proof with the proof that the small-step
semantics implement the big-step semantics to show that the instruction machine
implements the big-step semantics. In Section~\ref{sec:discussion} we discuss
threats to validity, future work, and related work. We conclude in
Section~\ref{sec:conclusion}. The Coq source code with all the definitions and
proofs described in this document is available at
\url{https://github.com/stelleg/cem\_coq}. 

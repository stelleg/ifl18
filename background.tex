\section{Background} \label{sec:background}

Programming languages can roughly be broken into two camps: those with
\emph{strict} and those with \emph{non-strict} semantics. A strict language is
one in which arguments at a call-site are always evaluated, while a non-strict
language only evaluates arguments when they are needed. One can further break
non-strict into two categories: call-by-name and call-by-need. Call-by-name is
essentially evaluation by substitution: an argument term or closure is
substituted for every instance of a corresponding variable. This has the
downside that it can result in exponential slowdown due to repeated work: every
variable dereference must re-evaluate the corresponding argument. Call-by-need
is an evaluation strategy devised to address this shortcoming. By sharing the
result of argument evaluation between instances of a variable, one avoids
duplicated work.  Unsurprisingly, call-by-need is the default semantics
implemented by compilers for non-strict languages like Haskell \cite{jonesstg}. 

Also perhaps unsurprisingly, call-by-need implementations tend to be more
complicated than their strict counterparts. For example, even attempts at simple
call-by-need abstract machines such as the Three Instruction Machine \cite{TIM}
require lambda lifting and shared indirections, both of which make formal
reasoning more difficult. Our $\mathcal{CE}$ machine avoids these complications
by using shared environments to share evaluation results between instances of a
variable. We showed that in addition to being simpler to implement and reason
about, performance of this approach can often compete with the state of the art
\cite{cem}. 

With recent improvements in higher order logics, machine verification of
algorithms has become a valuable tool in software development. Instead of
relying heavily on tests to reason about the correctness of programs,
verification can prove that algorithms implement their specification for
\emph{all} inputs, a very desirable property to have. Having the ability to
write down both the specification and the proof in a machine-checked logic
removes the vast majority of bugs found in hand-written proofs, ensuring far
higher confidence in correctness than other standard methods. This approach has
been confirmed to effectively remove all bugs by other methods as well, such as
fuzz testing \cite{yangfuzz}.

This approach applies particularly well to compilers. Often, the specification
for a compiler is relatively easy to write down: source level semantics for some
languages are exceedingly straightforward to specify. In addition, writing tests 
for compilers that cover all cases is even more hopeless than most domains, due
to the size of the domain and codomain. There is also a sense in which the
amortized return on investment is high: all reasoning about programs compiled
with a verified compiler is provably preserved. 

Due to the complexities discussed above involved in implementing lazy languages,
existing work has focused on compiling strict languages
\cite{chlipala2007certified, leroy2012compcert, cakeml14}. In this work we use
the simple $\mathcal{CE}$ machine as a base for a verified compiler of a lazy
language, using the Coq proof assistant. 

As with many areas of research, the devil is in the details. What exactly do
existing approaches mean when they claim their compilers are verified?
Essentially, a verified compiler of a functional language is one that preserves
computation of values. That is, we have a one directional implication that
\emph{if the source semantics computes a value, then the compiled code computes
an equivalent value} \cite{chlipala2007certified}. The important thing to note is that the
implication is only in one direction. If the source semantics never terminates,
this class of correctness theorem says nothing about the behavior of the
compiled code. This has consequences for Turing-complete source languages. If we
are unsure if a source program terminates, and wish to run it to check
experimentally if it does, if we run the compiled code and it returns a value,
we cannot be certain that it corresponds to a value computed in the source
semantics. 

While in theory one could solve this by proving the implication the other
direction, that is, \emph{if the compiled code computes a value then the source
semantics computes an equivalent value}, in practice this is prohibitively
difficult. Essentially, the induction rules for the abstract machine make
constructing such a proof monumentally tricky. 

One approach for getting around this issue is to try and capture the divergent
behavior by defining a diverging semantics explicitly \cite{functionalbigstep}.
Then we can safely say that \emph{if the source semantics diverges according to
our diverging semantics, then the compiled code also diverges}. Unfortunately,
this does not quite give us what we want, as we still cannot be sure that a value
computed by the compiled code corresponds to a value computed by the source
semantics. Essentially, the issue is that we do not have, and can't hope to have
a Turing-complete language, a constructive proof that every program evaluates to
a value or diverges.  

For this paper, we choose to take the approach of \cite{chlipala2007certified}
and define verification as the first implication above, focusing on the case in
which the source semantics evaluates to a value. This is still a very strong
result: any source program that has meaning compiles to an executable with
equivalent meaning. In addition, if we ever choose to extend the language with a
type system that ensures termination, or some notion of progress, then we can
use this proof in combination with our verification proof to prove the other
direction. 

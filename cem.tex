\section{\ce Big-Step Semantics} \label{sec:cem_big}

In this section we define our big-step source semantics. A big-step semantics
has the advantage of powerful, easy-to-use induction properties. This eases
reasoning about many program properties. We shall also define a small-step
semantics and prove that it implements the big-step semantics, but by showing
that our implementation preserves the big-step semantics, we prove preservation
of any inductive reasoning on the structure of evaluation tree.  

As discussed in Section~\ref{sec:introduction}, our source syntax is lambda calculus with
de Bruijn indices. De Bruijn indices count the number of intermediate lambdas
between them and their binding lambda.  

\begin{align*}
 t &::= t \; t \; | \; x \; | \;  \lambda \; t \\
 x &\in \mathbb{N}
\end{align*}

The essence of the \ce semantics is that we implement a shared
environment, and use its structure to share results of computations. This allows
for a simple abstract machine which can operate on lambda calculus directly,
which is uncommon among call-by-need abstract machines
\cite{jonesstg,launchburynatural,TIM,johnsson1984efficient}. This simplifies
formalization, as we do not need to prove that these transformations are
semantics-preserving. Another advantage to the \ce machine is that it
has constant sized closures, obviating the need to reason about re-allocating
the results of computation and adding indirections due to closure size changes
from thunk to value \cite{jonesstg}. We operate on closures, which combine terms with
pointers into the shared environment, which is implemented as a heap. Every heap
location contains a cell, which consists of a closure and a pointer to the next
environment location, which we will refer to as the environment continuation.
Variable dereferences index into this shared environment structure, and if/when
a dereferenced location evaluates to a value, the original closure (potentially
a thunk or closure not evaluated to WHNF) will be replaced with said value. The
binding of a new variable extends the shared environment structure with a new
cell. This occurs during application, which evaluates the left hand side to an
abstraction, then extend the environment with the argument term closed under the
environment pointer of the application. The App rule ensures that two
variables bound to the same argument closure will point to the same location in
the shared environment. Because they point to the same location by construction
of the shared environment, we can update that location with the value computed
at the first variable dereference, and then each subsequent dereference will
point to this value. The variable rule applies the update by indexing into the
shared environment structure and replacing the closure at that location with the
resulting value. The big-step semantics is described in the standard way by
logical relation in Figure~\ref{fig:bigstep}. It is worth noting that while the
closures in the heap cells are mutable, the shared environment structure is
never mutated. This property is crucial when reasoning about variable
dereferences. The $\mu\left(l, i\right)$ function looks up a variable index in
the shared environment structure by following environment continuation pointers,
returning the location and cell pointed to by the final step. Note that we
require that fresh heap locations are greater than zero. This is required for
reasoning about compilation to the instruction machine, which we will return to
in Section~\ref{sec:im_semantics}. While we constrain fresh heap locations to
not exists in the domain of the heap, for a real implementation, this is far too
strong a constraint, as it doesn't allow any sort of heap re-use. We return to
this issue in Section~\ref{sec:discussion}, and discuss how this could be
relaxed to either allow reasoning about garbage collection or direct heap-reuse.

\begin{figure}
\textbf{Syntax}
\begin{align*}
\tag{Term} t &::= i \; | \; \lambda t \; | \; t \; t  \\
\tag{Variable} i &\in \mathbb{N}  \\
\tag{Closure} c &::= t \left[l\right] \\
\tag{Value} v &::= \lambda t \left[l\right] \\
\tag{Heap} \mu &::= \epsilon \; | \; \mu \left[ l \mapsto \rho \right] \\
\tag{Environment} \rho &::= \bullet \; | \; c \cdot l \\
\tag{Location} l,f &\in \mathbb{N}  \\
\tag{Configuration} s &::= \left(c, \mu \right)
\end{align*}
\textbf{Semantics}
\begin{align*}
\tag{Id} \inference
{\mu \left( l, i \right) = l' \mapsto c \cdot l'' \quad 
 \left(c, \mu\right) \Downarrow \left(v, \mu'\right)}
{\left(i\left[l\right],\mu\right) \Downarrow \left(v, \mu'\left[l' \mapsto v \cdot l''\right]\right)}
\end{align*}
\begin{align*}
\tag{App} \inference
{\left(t\left[l\right], \mu\right) \Downarrow \left(\lambda t_2\left[l'\right], \mu'\right) 
   \quad f \not \in \textnormal{dom}\left(\mu'\right)
   \\ \left(t_2\left[f\right], \mu'\left[f \mapsto t_3\left[l\right] \cdot l'\right]\right)
         \Downarrow 
      \left(v, \mu'' \right) 
   }
{\left(t \; t_3\left[l\right], \mu\right) \Downarrow \left(v, \mu'' \right)}  
\end{align*}
\begin{align*}
\tag{Abs} \inference {} {\left(\lambda t\left[l\right], \mu\right) \Downarrow \left(\lambda t\left[l\right], \mu\right)}
\end{align*}
\caption{Big Step \ce Syntax and Semantics}
\label{fig:bigstep}
\end{figure}

The fact that our natural semantics is defined on lambda calculus with de Bruijn
indices differs from most existing definitions of call-by-need, such as
Ariola's call-by-need \cite{ariola1995call} or Launchbury's lazy semantics
\cite{launchburynatural}. These semantics are defined on lambda calculus with named
variables. While it should be possible to relate our semantics to these
\footnote{Both of these well known existing semantics have known problems that
arise during formalization, as discussed in Section~\ref{sec:discussion}}, the
comparison is certainly made more difficult by this disparity. A more fruitful
relation to semantics operating on lambda calculus with named variables would
likely be relating Curien's calculus of closures to call-by-name semantics
implemented with substitution. We return to this discussion in
Section~\ref{sec:discussion}.

As mentioned in Section~\ref{sec:introduction}, these big-step semantics do not
explicitly include a notion of nontermination. Instead, nontermination would be
implied by the negation of the existence of an evaluation relation. This
prevents reasoning directly about nontermination in an inductive way, but for
the purpose of our primary theorem this is acceptable. 

One interesting property of defining an inductive evaluation relation in a
language such as Coq is that we can do computation on the evaluation tree. In other
words, the evaluation relation given above defines a data type, one that we can
do computation on in standard ways. For example, we could potentially compute
properties such as size and depth, which would be related to operational properties
of compiled code. We hope in future work to explore this approach further.

Finally, given a term $t$, we define the initial configuration as
$\left(t\left[0\right], \epsilon\right)$. As discussed, the choice of the null
pointer for the environment pointer is not completely arbitrary, but chosen
across our semantics uniformly to represent failed environment lookup. 

\subsection{Call-By-Name}

In this section we define a call-by-name variant of our big-step semantics and
prove that it is an implementation of Curien's call-by-name calculus of
closures \cite{curien1991abstract}. 

See Figure~\ref{fig:bigstepname} for the definition of our call-by-name
semantics. Note that the only change from our call-by-need semantics is that we
do not update the heap location with the result of the dereferenced computation.
This is the essence of the difference between call-by-name and call-by-need.

A well known existing call-by-name semantics is Curien's calculus of closures
\cite{curien1991abstract}. Refer to Figure~\ref{fig:curien} for a formalization of this
semantics. This semantics defines closures as a term, environment pair, where an
environment is a list of closures. Abstractions are in weak head normal form,
variables index into the environment, and applications evaluate the left hand
side to a value, then extend the environment of the value with the closure of
the argument. 

We define a heterogeneous equivalence relation between our shared environment
and Curien's environment. Effectively, this relation is the proposition that
the shared environment structure is a linked list implementation of the
environment list in Curien's semantics. This is defined inductively, and we
require that every closure reachable in the environment is also equivalent.  We
say two closures are equivalent if their terms are identical and their
environments are equivalent. 

Given these definitions, we can prove that our call-by-name semantics implement
Curien's call by name semantics: 

\begin{theorem}
If a closure $c$ in Curien's call-by-name semantics is equivalent to a
configuration $c'$, and $c$ steps to $v$, then there exists a $v'$ that our
call-by-name semantics steps to from $c'$ that is equivalent to $v$.
\end{theorem}
\begin{proofoutline}
The proof proceeds by induction on Curien's step relation. The abstraction rule
is a trivial base case. The variable lookup rule uses a helper lemma that proves
by induction on the variable that if the two environments are equivalent and the
variable indexes to a closure, then the $\mu$ function will look up an
equivalent closure. The application rule uses a helper lemma proves that a fresh
allocation will keep any equivalent environments equivalent, and that the new
environment defined by the fresh allocation will be equivalent to the extended
environment of Curien's semantics.
\end{proofoutline}

\begin{figure}
\textbf{Syntax}
\begin{align*}
\tag{Term} t &::= i \; | \; \lambda t \; | \; t \; t  \\
\tag{Variable} i &\in \mathbb{N}  \\
\tag{Closure} c &::= t \left[\rho\right] \\
\tag{Value} v &::= \lambda t \left[\rho\right] \\
\tag{Environment} \rho &::= \bullet \; | \; c \cdot \rho \\
\end{align*}
\textbf{Semantics}
\begin{align*}
\tag{LEval}\inference
{t_1\left[\rho\right] {\Downarrow} \lambda t_2\left[\rho'\right] \\ 
 t_2\left[t_3\left[\rho\right] \cdot \rho'\right] \Downarrow v}
{t_1 t_3\left[\rho\right] \Downarrow v } 
\end{align*}
\begin{align*}
\inference
{c_i \Downarrow v}
{\tag{LVar} i \left[c_0 \cdot c_1 \cdot ... c_i \cdot \rho\right] \Downarrow v}
\end{align*}
\caption{Curien's call-by-name calculus of closures}
\label{fig:curien}
\end{figure}

\begin{figure}
\textbf{Syntax}
\begin{align*}
\tag{Term} t &::= i \; | \; \lambda t \; | \; t \; t  \\
\tag{Variable} i &\in \mathbb{N}  \\
\tag{Closure} c &::= t \left[l\right] \\
\tag{Value} v &::= \lambda t \left[l\right] \\
\tag{Heap} \mu &::= \epsilon \; | \; \mu \left[ l \mapsto \rho \right] \\
\tag{Environment} \rho &::= \bullet \; | \; c \cdot l \\
\tag{Location} l,f &\in \mathbb{N}  \\
\tag{Configuration} s &::= \left(c, \mu \right)
\end{align*}
\textbf{Semantics}
\begin{align*}
\tag{Id} \inference
{\mu \left( l, i \right) = l' \mapsto c \cdot l'' \quad 
 \left(c, \mu\right) \Downarrow \left(v, \mu'\right)}
{\left(i\left[l\right],\mu\right) \Downarrow \left(v, \mu'\right)}
\end{align*}
\begin{align*}
\tag{App} \inference
{\left(t\left[l\right], \mu\right) \Downarrow \left(\lambda t_2\left[l'\right], \mu'\right) 
   \quad f \not \in \textnormal{dom}\left(\mu'\right)
   \\ \left(t_2\left[f\right], \mu'\left[f \mapsto t_3\left[l\right] \cdot l'\right]\right)
         \Downarrow 
      \left(v, \mu'' \right) 
   }
{\left(t \; t_3\left[l\right], \mu\right) \Downarrow \left(v, \mu'' \right)}  
\end{align*}
\begin{align*}
\tag{Abs} \inference {} {\left(\lambda t\left[l\right], \mu\right) \Downarrow \left(\lambda t\left[l\right], \mu\right)}
\end{align*}
\caption{Big Step Call-by-Name \ce Syntax and Semantics}
\label{fig:bigstepname}
\end{figure}

By proving that Curien's semantics is implemented by the call-by-name variant of
our semantics, we provide further evidence that our call-by-need is a
meaningful semantics. While eventually we would like to prove that the
call-by-need semantics implements an optimization of the call-by-name, we leave
that for future work.

One important note is that nowhere do we require that a term being evaluated is
closed under its environment. Indeed, it's possible that a term with free variables
can be evaluated by both semantics to a value as long as a free variable is
never dereferenced. This theme will recur through the rest of the paper, so it
is worth keeping in mind.  

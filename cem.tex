\section{Source Semantics} \label{cem_big}

In this section we define with our big step source semantics. Big step semantics
have the advantage of powerful, easy to use induction properties. This eases
reasoning about many program properties. We'll also define a small step
semantics and prove that it implements the big step semantics, but by showing
that our implementation preserves the big-step semantics, we prove preservation
of any inductive reasoning on the structure of evaluation tree.  

As discussed in Section~\ref{intro}, our source syntax is lambda calculus with
deBruijn indices. deBruijn indices count the number of intermediate lambdas
between them and their binding lambda.  
\begin{align*}
 t &::= t \; t \; | \; x \; | \;  \lambda \; t \\
 x &\in \mathbb{N}
\end{align*}
The essence of the $\mathcal{CE}$ semantics is that we implement a shared
environment, and use its structure to share results of computations. This allows
for a simple abstract machine which can operate on lambda calculus directly
which is uncommon among call-by-need abstract machines
\cite{stg,launchbury,TIM}. This simplifies formalization, as we don't need to
prove that these transformations are semantics-preserving. Another advantage to
the $\mathcal{CE}$ machine is that it has constant sized closures, preventing
the need to reason about re-allocating the results of computation and adding
indirections due to closure size changes from thunk to value \cite{stg}. We
operate on closures, which combine terms with pointers into the shared
environment, which is implemented as a heap. Every heap location points to a
cell, which contains a closure and a pointer to the next environment location if
necessary, which we will refer to as the environment continuation. Variable
dereferences index into this shared environment structure, and if/when a
dereferenced location evaluates to a value, the original closure (potentially a
thunk or closure not evaluated to WHNF) will be replaced with said value. The
binding of a new variable extends the shared environment structure with a new
cell. This is done by application, which evaluate the left hand side to an
abstraction, then extend the environment with the argument term closed under the
environment pointer at the environment of the application. The application rule
enforces the fact that two variables bound to the same argument closure will
point to the same location in the shared environment. Because they point to the
same location by construction of the shared environment, we can update that
location with the value computed at the first variable dereference, and then
each subsequent dereference will point to this value. The variable rule applies
the update by indexing into the shared environment structure and replacing the
closure at that location with the resulting value. The big step semantics are
described in the standard way by logical relation in Figure~\ref{fig:bigstep}. A
few notes about definitions used in the semantics: the \texttt{update $\Psi$ x
v} function replaces the closure at location \texttt{x} in heap $\Psi$ with
value \texttt{v}. The curly brackets construct a heap cell containing a closure
and a environment continuation. The \texttt{app} constructor represents an
application, while the \texttt{var} constructor is a variable constructor and
$\lambda$ is the constructor for an abstraction. The \texttt{clu} function looks
up a variable index in the shared environment structure by following environment
continuation pointers. The $\langle \Phi \rangle \texttt{c}$ notation constructs
a configuration from a heap $\Phi$ and a closure \texttt{c}. Note that we
require that fresh heap locations are greater than zero. This is required for
reasoning about compilation to the instruction machine, which we will return to
in Section~\ref{sec:im}. \texttt{isfresh} for this paper is defined as $f \notin
dom \big( \Psi \big)$. Of course, for a real implementation, this is far too strong a
constraint, as it doesn't allow any sort of heap re-use. We return to this issue
in Section~\ref{sec:discussion}, and discuss how this could be relaxed to either
allow reasoning about garbage collection or direct heap-reuse.

\begin{figure}
\begin{lstlisting}
Inductive step : conf -> conf -> Type :=
  | Id : $\forall$ M x y z $\Phi$ $\Psi$ v e, 
              clu y e $\Phi$ = Some (x, {M, z}) -> 
                    $\langle$$\Phi$$\rangle$M $\Downarrow$ $\langle$$\Psi$$\rangle$v ->
      $\langle$$\Phi$$\rangle$close (var y) e $\Downarrow$ $\langle$update $\Psi$ x v$\rangle$v
  | Abs : $\forall$ N $\Phi$ e, 
          $\langle$$\Phi$$\rangle$close ($\lambda$N) e $\Downarrow$ $\langle$$\Phi$$\rangle$close ($\lambda$N) e
  | App : $\forall$ N M B B' $\Phi$ $\Psi$ $\Upsilon$ f e ne ae, 
                      isfresh (domain $\Psi$) f -> 
                            f > 0 ->
          $\langle$$\Phi$$\rangle$close M e $\Downarrow$ $\langle$$\Psi$$\rangle$close ($\lambda$B) ne -> 
      $\langle$$\Psi$, f ↦ {close N e, ne}$\rangle$close B f $\Downarrow$ $\langle$$\Upsilon$$\rangle$close ($\lambda$B') ae   ->
              $\langle$$\Phi$$\rangle$close (app M N) e $\Downarrow$ $\langle$$\Upsilon$$\rangle$close ($\lambda$B') ae
where " c1 '$\Downarrow$' c2 " := (step c1 c2).
\end{lstlisting}
\caption{Big Step $\mathcal{CE}$ Semantics}
\label{fig:bigstep}
\end{figure}

In addition to the above description there are some important properties to
notice for the purpose of verification. One is that we specify that in addition
to requiring that a heap location for a new bound closure is fresh, we require
that the it is positive. While not necessary for the correctness of the
semantics, this will be useful later when defining a relation between the heap
described here and the machine code heap, ensuring that we don't artificially
extend the empty environment. It's also worth noting that while the closures in
the heap cells are mutable, the shared environment structure is never mutated.
This property is crucial when reasoning about variable dereferences. 

The fact that our natural semantics are defined on lambda calculus with deBruijn
tindices differs from most existing definitions of call-by-need, such as
Ariola's call-by-need \cite{ariola} or Launchbury's lazy semantics
\cite{launchbury}.  These semantics are defined on lambda calculus with named
variables. While it should be possible to relate our semantics to these
\footnote{Both of these well known existing semantics have known bugs, as
discussed in Section~\ref{discussion}}, it is certainly made more difficult by
this disparity. A more fruitful relation to semantics operating on lambda
calculus with named variables would likely be relating curiens calculus of
closures to call-by-name semanticss implemented with substitution. We'll return
to this discussion in Section~\ref{discussion}.

As mentioned in Section~\ref{intro}, these big step semantics don't explicitly
include a notion of nontermination. Instead, nontermination would be implied by
the negation of the existence of an evaluation relation. This prevents reasoning
directly about nontermination in an inductive way, but for our the purpose of
our primary lemma this is fine. 

One interesting property of defining an inductive evaluation relation in a
language like Coq is that we can do computation on the evaluation tree. In other
words, the evaluation relation given above defines a data type, one that we can
do computation on in standard ways. For example, we could potentially compute
properties like size and depth, which would be related to operational properties
of compiled code. We hope in future work to explore this approach further.

Finally, given a term $t$, we define the initial configuratoin as $\langle [] \rangle
\texttt{close t 0}$. As discussed, the choice of the null pointer for the envioronment
pointer is not arbitrary, but chosen across our semantics uniformly to represent
failed environment lookup. 

\subsection{Call-By-Name}

In this section we define a call-by-name variant of our big step semantics and
prove that it is an implementation of Curien's call-by-name calculus of
closures \cite{curien}. 

See Figure~\ref{fig:bigstepname} for the definition of our call-by-name
semantics. Note that the only change from our call-by-need semantics is that we
don't update the heap location with the result of the dereferenced computation.
This is the essence of the difference between call-by-name and call-by-need.

A well known existing call-by-name semantics is Curien's calculus of closures
\cite{curien}. Refer to Figure~\ref{fig:curien} for a formalization of this
semantics. This semantics defines closures as a term, environment pair, where an
environment is a list of closures. Abstractions are in weak head normal form,
variables index into the environment using the \texttt{nth\_error} standard
library function, and applications evaluate the left hand side to a value, then
extend the environment of the value with the closure of the argument. 

We define a heterogeneous equivalence relation between our shared environment
structure and Curien's structure. Effectively, this relation is the claim that
the shared environment structure is a linked list implementation of the
environment list in Curien's semantics. This is defined inductively, and we
require that every closure reachable in the environment is also equivalent.  We
say two closures are equivalent if their terms are identical and their
environments are equivalent. 

Given these definitions, we can prove that our call-by-name semantics implement
Curien's call by name semantics: 

\begin{theorem}
If a closure $c$ in Curien's call-by-name semantics is equivalent to a
configuration $c'$, and $c$ steps to $v$, then there exists a $v'$ that our
call-by-name semantics steps to from $c'$ that is equivalent to $v$.
\end{theorem}
\begin{proofoutline}
The proof proceeds by induction on Curien's step relation. The abstraction rule
is a trivial base case. The variable lookup rule uses a helper lemma that proves
by induction on the variable that if the two environments are equivalent and the
\texttt{nth\_error} function looks up a closure, then the \texttt{clu} function
will look up an equivalent closure. The application rule uses a helper lemma
proves that a fresh allocation will keep any equivalent environments equivalent,
and that the new environment defined by the fresh allocation will be equivalent
to the extended environment of Curien's semantics.
\end{proofoutline}

\begin{figure}
\begin{lstlisting}
Inductive closure := | close : tm -> list closure -> closure. 
Definition env := list closure.

Inductive step : closure -> closure -> Type := 
  | Abs : $\forall$ b e, step (close ($\lambda$b) e) (close ($\lambda$b) e)
  | Var : $\forall$ x e v c, nth_error e x = Some c -> step c v -> step (close (var x) e) v
  | App : $\forall$ m n b e v mve,
      step (close m e) (close ($\lambda$b) mve) ->
      step (close b (close n e::mve)) v ->
      step (close (app m n) e) v.
\end{lstlisting}
\caption{Curien's Calculus of Closures}
\label{fig:curien}
\end{figure}


\begin{figure}
\begin{lstlisting}
Inductive step : conf -> conf -> Type :=
  | Id : $\forall$ M x y z $\Phi$ $\Psi$ v e, 
          clu y e $\Phi$ = Some (x, {M, z}) -> 
                    $\langle$$\Phi$$\rangle$M $\Downarrow$ $\langle$$\Psi$$\rangle$v ->
            $\langle$$\Phi$$\rangle$close (var y) e $\Downarrow$ $\langle$$\Psi$$\rangle$v
  | Abs : $\forall$ N $\Phi$ e, 
        $\langle$$\Phi$$\rangle$close ($\lambda$N) e $\Downarrow$ $\langle$$\Phi$$\rangle$close ($\lambda$N) e
  | App : $\forall$ N M B B' $\Phi$ $\Psi$ $\Upsilon$ f e ne ae, 
              isfresh (domain $\Psi$) f -> 
                            f > 0 ->
          $\langle$$\Phi$$\rangle$close M e $\Downarrow$ $\langle$$\Psi$$\rangle$close ($\lambda$B) ne -> 
      $\langle$$\Psi$, f ↦ {close N e, ne}$\rangle$close B f $\Downarrow$ $\langle$$\Upsilon$$\rangle$close ($\lambda$B') ae   ->
              $\langle$$\Phi$$\rangle$close (app M N) e $\Downarrow$ $\langle$$\Upsilon$$\rangle$close ($\lambda$B') ae
where " c1 '$\Downarrow$' c2 " := (step c1 c2).
\end{lstlisting}
\caption{Call-by-Name $\mathcal{CE}$ Semantics}
\label{fig:bigstepname}
\end{figure}

By proving that Curien's semantics is implemented by the call-by-name variant of
our semantics, we provide further evidence that our call-by-need is a
meaningful semantics. While eventually we'd like to prove that the call-by-need
semantics implement an optimization of the call-by-name, we leave that for
future work.

One imporant note is that nowhere do we require that a term being evaluated is
complete closed by its environment. Indeed, we don't and it's possible that a
term with free variables can be evaluated by both semantics. This theme will
recur through the rest of the paper, so it's worth keeping in mind.  

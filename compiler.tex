\section{Compiler} \label{sec:compiler}

In this section we describe the and the compiler, which compiles lambda terms to
programs. The program recurses on lambda terms, keeping a current index into the
program to ensure correct linking without a separate pass. For variables, when
we get to zero we push the current environment pointer and a null instruction
pointer to denote the update marker to the location of the closure being
entered. Then we \texttt{mov} the closure at that location into \texttt{IP} and
\texttt{EP}, recalling that the \texttt{jump} sets the \texttt{IP}. For the
nonzero variables, we replicate traversing the environment pointer \texttt{i}
times before loading the closure. For applications, we calculate the program
location of the argument basic block, and push that and the current environment
pointer onto the stack, effectively pushing an argument closure on top of the
stack. We then jump to the left hand side of the application, as is standard for
push-enter evaluation. For abstractions, we choose to use a conditional jump
depending on whether the top of the stack is a null pointer (and therefore an
update marker) or a valid instruction pointer (and therefore an argument). If it
is an update marker, we update the heap location defined by the update marker
with the current value instruction pointer and the current environment pointer.
Note that we must point to the first value basic block, as this value could
later update another heap location as well. In the case that the top of the
stack was a valid instruction pointer, we allocate a new chunk of 3 word of
memory, and mov the argument closure into it, with the current environment
pointer as the environment continuation. We then set our current environment
pointer to this fresh location. This is the process by which we extend our
shared environment structure in the instruction machine. Finally, we perform an
unconditional jump to the next basic block, which is the first basic block of
the compiled body of the lambda. As this is an unconditional jump to the next
basic block, for real machine code this jump can be omitted. 

\begin{figure}
\begin{lstlisting}
Fixpoint var_inst (i : nat) : BasicBlock := match i with
  | 0 => push EP ;
         push (RC 0) ;
         mov (EP%0) R1 ;
         mov (EP%1) EP ;
         jump None R1
  | S i => mov (EP%2) EP ;
           var_inst i
  end.

Fixpoint compile (t : tm) (k : nat) : Program := match t with
  | var v => [var_inst v]
  | app m n => let ms := compile m (1+k) in
               let nk := 1+k+length ms in
                push EP ;
                push (RC nk) ;
                jump None (RC (1+k)) ::
                ms ++
                compile n nk
  | lam b => pop R1 ;
             jump (Some (RW (WR R1), (1+k))) (RC (2+k)) ::
             (*Update*)
             pop R1 ;
             mov (RC k) (R1%0) ;
             mov EP (R1%1) ;
             jump None (RC k) ::
             (*Take*)
             new 3 R2 ;
             mov R1 (R2%0);
             pop (R2%1) ;
             mov EP (R2%2) ;
             mov R2 EP ;
             jump None (3+k) ::
             compile b (3+k)
  end. 
\end{lstlisting}
\caption{Compiler Definition}
\label{fig:compiler}
\end{figure}

Being able to define the full compiler this simply is what enabled this
verification project. Other, more sophisticated implementations of call-by-need,
such as the STG machine, are much harder to implement and reason about. It is
worth noting that despite its simplicity, initial tests imply that performance
is not as horrible as one might suspect, and is often competitive with state of
the art \cite{cem}.

As with the relation discussed in Section~\ref{sec:cem}, note that even when
compiling, we do not require that a term is closed to compile it. Indeed, we
will generate code that if entered is nonsense. Because we are only concerned
with proving that we implement the source semantics in the case that it
evaluates to a value, this is not a problem. If we wanted to strengthen our
proof further, to show that if the source semantics gets stuck trying to
dereference a free variable, the implementation would get stuck in the same way,
with both failing to dereference a null pointer.    

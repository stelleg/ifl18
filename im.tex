\section{Instruction Machine} \label{sec:im_semantics}

Here we describe in full the instruction machine syntax and semantics. We
choose a simple stack machine with a Harvard architecture (with separate
instruction and heap memory). We use natural numbers for pointers, though it
shouldn't be too difficult to replace these with standard-sized machine words,
e.g., 64 bits, making the stack and malloc operations partial. Our stack is
represented as a list of pointers, though again it should be a relatively
straightforward exercise to represent the stack in main memory. With the fixed
machine word size, we would need to make push operations partial to represent
stacks this way. We define our machine to have only four registers: an
instruction pointer, an environment pointer, and two scratch registers. Our
instruction set is minimal, consisting only of a conditional jump instruction,
pop and push instructions, a mov instruction, and a new instruction for
allocating new memory. Note that for our program memory, we have pointers to
basic blocks, but for simplicity of proofs we choose to not increment the
instruction pointer within a basic block.  Instead, the instruction pointer is
constant within a basic block, only changing between basic blocks. In fact, we
represent the program as a list of basic blocks, with pointers indexing into the
list. This has the advantage of letting us easily reason about sublists and
their relation to terms. As with other design decisions, this also should be
fairly unproblematic for formalization to a more realistic hardware design. The
full syntax of the machine is given in Figure~\ref{fig:im_syntax}. Note that
curly brackets $\left\{\right\}$ denote optionality, while stars $*$ denote zero
or more elements, represented as a list. Note that we'll use some common list
terminology, such as bracket notation for indexing, i.e. $l\left[i\right]$ accesses the
$i$'th element in $l$ (we don't worry about the partiality in this presentation
of this operation, see the Coq implementation for a full treatment). We also use
$\concat$ for list concatenation, and $::$ for consing an element onto the head
of a list. 

\begin{figure}
\begin{align}
n, l, w &\in \mathbb{N} \tag{Machine Word} \\
r &:= ip \; | \; ep \; | \; r1 \; | \; r2 \tag{Registers} \\
wo &:= r \; | \; r \% n \tag{Write Operands} \\
ro &:= wo \; | \; n \tag{Read Operands} \\
i &:= push \; ro \; | \; pop \; wo \; | \; new \; n \; wo \; | \; mov \; ro \; wo \tag{Instructions} \\
bb &:= i : bb \; | \; jump \left\{ro, l\right\} ro \tag{Basic Block} \\
p &:= bb* \tag{Program} \\
s &:= w* \tag{Stack} \\
h &:= \left(l, w\right)* \tag{Heap}\\
S &:= \langle rf, p, s, h \rangle \tag{State} 
\end{align}
\caption{Instruction Machine Syntax}
\label{fig:im_syntax}
\end{figure}

We separate read (ro) and write (wo) operands. Write operands can be registers
or memory (defined by a register and a constant offset). Read operands can be
any write operand or a constant. For reading, we have a read relation, which
takes a read operand and a state and is inhabited when the third argument can be
read from that read operand in that state. Similarly, a write relation is
inhabited when writing the second argument into the first in a state defined by
the third argument results in the state defined by the fourth argument.  

The machine semantics should be fairly unsurprising. A State, consists of a
register file, program memory, a stack, and a heap. The push instruction takes a
read operand and pushes it onto the stack. The pop instruction pops the
top of the stack into a write operand. The mov instruction moves a machine word
from a read operand to a write operand. The jump instruction is parameterized by
an optional pair, which, if present, reads the first element of the pair from a
read operand, checks if it is zero, and if so sets the IP to the second element
of the pair, which is a constant pointer. If the condition is not zero, then it
sets the IP to the instruction pointer contained in the second jump argument. If
we pass nothing as the first argument, then it becomes an unconditional set of
the IP to the value read from the second argument.  Note that the second
argument is a read operand, so it can either be a constant or read from a register
or memory. This means it can be effectively either a direct or indirect jump,
both of which are used in the compilation of lambda terms. The new instruction
allocates a contiguous block of new memory and writes the resulting pointer to
the fresh memory into a write operand. We take the approach of not choosing a
particular allocation strategy. Instead, we follow existing approaches and
parameterize our proof on the existence of such functionality
\cite{chlipala2007certified}. For simplicity, we assume that the allocation
function returns completely fresh memory, though it should be possible to modify
this assumption to be less restrictive, i.e., let it re-use heap locations that
are no longer live. The complete semantics of the machine is given in
Figure~\ref{fig:im_semantics}.  Note that we separate instruction steps and
basic block steps. Recall that a basic block is a sequence of instructions that
ends with a jump. The Step BB relation will execute the instructions
in the basic block in order, then set the IP in accordance with the
jump semantics. The Step relation dereferences a basic block at the current IP,
and if executing the basic block results in a new state, then the machine
executes to that state. 

\begin{figure}
\begin{gather*}
\tag{Push}
\inference{
read \; ro \; \langle rf, p s, h \rangle \; v \\
bb, \langle rf, p, v::s, h \rangle \rightarrow_{bb} S 
}{push \; ro : bb, \langle rf, p, s, h \rangle \rightarrow_{bb} S} \\ \\
\tag{Pop}
\inference{
write \; wo \; w \langle rf, p, s, h \rangle S' \\
bb, S' \rightarrow_{bb} S \\
}{pop \; wo:bb, \langle rf, p, w::s, h \rangle \rightarrow_{bb} S}  \\ \\
\tag{New} \inference{
\forall i < n, f+i \notin \textsf{dom}\left(h\right) \\
write \; wo \; f \langle rf, p, s, zeroes \; n \; f \concat h \rangle S' \\
bb, S' \rightarrow_{bb} S
}{new \; n \; wo:bb, \langle rf, p, s, h \rangle \rightarrow_{bb} S} \\ \\
\tag{Mov} \inference{
read \; ro \; s \; v \quad
write \; wo \; v \; S \; S' \quad
bb, S' \rightarrow_{bb} S''
}{mov \; ro \; wo:bb, S \rightarrow_{bb} S''} \\ \\
\tag{Jump 0} \inference{
read \; ro \; S \; 0 \quad
write \; ip \; k \; S \; S'
}{jump \; \left(ro, k\right) \; j, S \rightarrow_{bb} S'} \\ \\
\tag{Jump S} \inference{
l > 0 \quad 
read \; ro \; S \; l \\
read \; j \; S \; k \quad
write \; ip \; k \; S \; S'
}{jump \; \left(ro, k'\right) \; j, S \rightarrow_{bb} S'} \\ \\
\tag{Jump} \inference{
read \; ro \; S \; l \quad
write \; ip \; l \; S \; S'
}{jump \; ro:S \rightarrow_{bb} S'} \\ \\
\tag{Enter} \inference{
read \; ip \; \langle rf, p, s, h \rangle \; k \\
p\left[k\right] = bb \\
bb, \langle rf, p, s, h \rangle \rightarrow_{bb} S' 
}{\langle rf, p, s, h \rangle \rightarrow S'}
\end{gather*}
\caption{Instruction Machine Semantics}
\label{fig:im_semantics}
\end{figure}


\section{Compiler Correctness} \label{sec:correctness}

In this section we define a relation between the state of the small-step
semantics and the state of the instruction machine semantics, and show that the
instruction machine implements the small-step semantics under that relation. 

In general, we implement closures as instruction pointer, environment pointer
pairs. For the instruction pointers, we relate them to terms via the compile
function defined in Section~\ref{sec:compiler}. Essentially, we require that the
instruction pointer points to a list of basic blocks that the related term
compiles to. For the current closure, we relate the instruction pointer register
in the instruction machine to the current term in the small-step source
semantics. The environment pointers of each machine are more similar. Given a
relation between the heaps of the two machines, we define the relation between
two environment pointers as existing in the relation of the heaps, or both being
the null pointers. While it should be possible to avoid this special case,
during the proof it became apparent that not having the special case made the
proof significantly harder. This forces us to add the constraint to all machines
that pointers are non-null, which for real hardware shouldn't be an issue.  

We use null pointers in two crucial ways. One is to explicitly define the root
of the shared environment structure in both the source semantics and the machine
semantics. The other use is for instruction pointers. To differentiate between update
markers and pointers to basic blocks, we use a null pointer to refer to an
update marker, and a non-null pointer as an
instruction pointer for an argument closure. Note that in fact, while the null
pointers in heaps required us to only allocate non-null fresh locations in the
heaps of our semantics, using null pointers to denote update markers requires no
change to our program generation, due to the fact that an argument term of an
application cannot occur at position 0 in the program.

The relation between the heaps of the small-step source semantics and the
instruction machine is the trickiest part of the state relation. Note that for
each location in the source semantics heap, we have a cell with a closure and
environment continuation pointer. Naturally, the instruction machine represents
these as three pointers: two for the closure (the instruction pointer and
environment pointer) and one for the environment continuation. The easiest
approach turned out to be to use the structure of the heap constructs to define a
one-to-three mapping between this single cell and the three machine words. The
structure used for each of the heaps is a list of pointer, value bindings.
We use the ordering of these bindings in the list to define a one binding to
three binding mapping between the source heap and the machine heap. We define 
a membership relation that defines when an element is in our heap relation,
proceeding recursively on the inductive relation structure. This allows us to
define a notion of which pairs of each type of closure are in the heap, along
with their respective locations. Due to the ordering in which they are allocated
in the heap during evaluation, each pair of memory allocations corresponds to an
equivalent cell. We use this property as a heap equivalence property that is
preserved through evaluation: every binding pair in the heap relation property
described above defines equivalent closures and environment continuations. For
the relation between our stacks, we define a similar notion.  For update
markers, we require that every update marker points to related environments
(they are two pointers that exist in the heap relation). For argument closures,
we require that the closures are equivalent (the instruction pointer and
environment pointer are equivalent to their respective counterparts in the
small-step semantics). 

In summary, we require that the current closure in the small-step semantics is
equivalent to the closure represented by the instruction pointer, environment
pointer pair, and that the stacks and the heaps are equivalent. The actual Coq
implementation of this relation is too involved to relate directly here. We
encourage the reader to read the linked Coq source to fully appreciate it.

Given this relation between heaps, we can state our primary lemma.
\begin{lemma} \label{lem:cesm_im}
Given that an instruction machine state $i$ is related to a small-step
semantics state $s$, and that small-step semantics state steps to a new state
$s'$, the instruction machine will step in zero or more steps to a related state
$i'$.
\end{lemma}
\begin{proofoutline}
Our proof proceeds by case analysis on the step rules for the small-step
semantics. We'll focus on the second half of the proof, that $i'$ is related to
$s'$. The proofs that $i$ evaluates to $i'$ follow fairly directly from the
compiler definition given in Section~\ref{sec:compiler}. For the Var rule,
because we need to proceed by induction, we have to define a separate lemma and
proceed by induction on a basic block while forgetting the program, as the
induction hypothesis is invalid in the presence of the program. We then use the
lemma to show that evaluation of a compiled variable implements the evaluation
of the variable in the small-step semantics. In particular, we use the null
environment as a base case for our induction, as we know the only way lookup
could fail is if both environment pointers are null, but that cannot be the
case due to the fact that we know that the small-step semantics must have
successfully looked up its environment pointer in the heap. Therefore the only
option is for both environment pointers to exist in the heap relation, which
when combined with the heap equivalence relation in the outer proof gives us the
necessary property that the environment continuations are equivalent. Finally,
because the last locations reached must have been in the heap relation, we know
they are equivalent environment pointers, and therefore the stack relation is
preserved when we push the update marker onto the heap. For the App rule, we use
the definition of our compiler to prove that the argument term and argument
instruction pointer are equivalent and that the left hand side term and
instruction pointer are also equivalent. They share an environment pointer which
is equivalent by the fact that the application closures are related.  This
proves that the stack relation is preserved as well as the current closure,
while the heap is unchanged. For the Lam rule, we allocate a fresh variable and
because of our stack relation we can be sure that the closures that we
allocate are equivalent, as well as the environment continuations, as they are
taken from the previous current continuation. Because of how we define it, the
new allocations are equivalent under our heap relation, and preserve heap
equivalence. Finally, the Upd rule trivially preserves the stack and current
closure relations, and for proving that the relation is preserved for the
heap, we proceed by induction on the heap relation. In addition, we must prove
a supporting lemma that all environment relations are preserved by the update.
\end{proofoutline}

We now have a proof that the small-step semantics implements the big-step
semantics, and a proof that the instruction machine implements the small-step
semantics. We can now combine these to get our correct compiler theorem. 

\begin{theorem} 
\label{thm:correctness}
If a term $t$ placed into the initial configuration for the big-step semantics
evaluates to a value configuration $v$, then the instruction machine starting
in the initial state with \texttt{compile 0 t} as its program will evaluate to a
related state $v'$.  
\end{theorem}
\begin{proofoutline}
We first require that the relation defined between the small-step semantics
state and the instruction machine state holds for the initial configurations.
This follows fairly directly from the definition of the initial conditions and
the compile function. Second, we have by definition of reflexive transitive
closure that Lemma~\ref{lem:cesm_im} implies that if the reflexive transitive
closure of the small-step relation evaluates in zero or more steps from a
state $c$ to a state $v$, then a related state of the instruction machine $c'$
will evaluate to a state $v'$ which is related to $v$. We use these two facts,
along with the proof that the small-step implements the big-step for any stack,
specialized on the empty stack, to prove our theorem. 
\end{proofoutline}

It is worth recalling exactly what the relation implies about the two value
states. Namely, in addition to the value closures being equivalent, their heaps
and environments are equivalent, so that every reachable closure in the
environment is equivalent between the two.


